\chapter{Virtual Memory I}

\section{Motivation: Problems with Memory}
\begin{enumerate}
    \item Not enough space
    \item Holes in address space (memory fragmentation)
    \item Programs may crash/corrupt if multiple programs access the same memory address
\end{enumerate}

\section{Virtual Memory}
Virtual memory takes program addresses and maps them to physical RAM addresses. The program only sees the virtual address. Virtual memory helps enforce isolation: each program thinks it's running by itself.

\medskip

How does a program access memory?
\begin{enumerate}
    \item Program executes a load specifying a virtual address (VA)
    \item Computer translates the address to the physical address (PA) in memory
    \item If the PA is not in memory, the OS loads it from disk
    \item The computer reads the RAM using the PA and returns the data to the program
\end{enumerate}

\section{Page Tables}
Page table is the map from VA to PA. Suppose we have one entry for every word in our address space, we would have $2^{32}$ (total memory) $2^2$ (byte/word) $= 2^{30}$ entries. We want to reduce the number of entries in the page table by dividing memory into chunks (\emph{pages}) rather than words. A virtual page number (VPN) maps to a physical page number (PPN).

\subsection{Address Translation}
Both VA and PA store their respective page numbers and a page offset.

\subsection{Page Faults}
What happens when a page is not in RAM?
\begin{enumerate}
    \item Hardware (CPU) generates a \emph{page fault exception}
    \item The hardware jumps to the OS page fault handler to fix it
    \item The OS chooses a page to evict from RAM (if no more free space)
    \item If the page is dirty, it needs to be written back to disk first
    \item The OS updates the corresponding page table entry (PTE) to point to disk
    \item The OS brings in the page we wanted disk and puts it in RAM
    \item The OS updates the PTE of the new page
    \item The OS jumps back to the instruction that caused the page fault (This
time it won’t cause a page fault since the page has been loaded.)
\end{enumerate}

\subsection{Page Replacement Policies}
Generally, there is a tradeoff between performance and cost (metadata).
\begin{itemize}
    \item First in, first out (FIFO)
    \item Least recently used (LRU)
    \item Random
\end{itemize}

\subsection{Page Table Metadata}
\begin{description}
    \item[Valid:] 1 if page is in RAM and mapping is valid
    \item[Dirty:] 1 if page on RAM is more up to date than page on disk
    \item[Permission Bits:] for read, write, execute
\end{description}
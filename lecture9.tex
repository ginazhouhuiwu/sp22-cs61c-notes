\chapter{Compiler, Assembler, Linker, Loader (CALL)}

\section{Interpretation vs Translation}
We can either \emph{interpret} a high-level language when efficiency is not critical or \emph{translate} to a lower-level language to increase performance.
\begin{itemize}
    \item Translated/compiled code is more efficient and helps hide the program source from users.
\end{itemize}

\section{Compiler}
Compiler converts a high-level language file into an assembly language file.
\begin{itemize}
    \item Input: High-level language code (\texttt{foo.c})
    \item Output: Assembly language code (\texttt{foo.s})
\end{itemize}

\subsection{Compiler Steps}
\begin{description}
    \item[Lexer:] Tokenizes inputs
    \item[Parser:] Turns tokens into an abstract syntax tree
    \item[Semantic Analysis and Optimization:] Checks for semantic errors
    \item[Code Generation:] Outputs assembly code
\end{description}

\section{Assembler}
Assembler removes pseudo-instructions, converts what it can to machine language, resolves relative addressing, and creates a checklist for the linker (relocation table).
\begin{itemize}
    \item Input: Assembly language code (\texttt{foo.s})
    \item Output: Object code (\texttt{foo.o})
\end{itemize}

\subsection{Assembler Directives}
Gives directions to assembler but do not produce machine instructions.
\begin{description}
    \item[\texttt{.text}:] Subsequent items put in user text segment (machine code)
    \item[\texttt{.data}:] Subsequent items put in user data segment (binary rep of data in source file)
    \item[\texttt{.global sym}:] Declares \texttt{sym} global and can be referenced from other files
    \item[\texttt{.string str}:] Store the string \texttt{str} in memory and null-terminate it
    \item[\texttt{.word w1...wn}:] Store the $n$ 32-bit quantities in successive memory words
\end{description}

\subsection{Producing Machine Language}
Simple cases such as arithmetic have all the information stored within 32-bit instructions. But how are PC-relative branches and jumps handled?
\begin{description}
    \item[Forward Reference Problem:] Branch instructions may refer to unseen labels $\implies$ solve using two passes.
    \begin{itemize}
        \item First pass: remembers positions of labels
        \item Second pass: uses label positions to generate code
    \end{itemize}
    \item[Jumps:] Count number of instructions between target and jump to determine offset (\emph{position-independent code}).
    \item[References to Static Data:] Cannot be determined yet, so we create 2 tables:
    \begin{enumerate}
        \item Symbol Table: items that may be used by other files (exportable)
        \begin{itemize}
            \item labels: function calling
            \item data
        \end{itemize}
        \item Relocation Table: items whose address this files need, which that linker will replace later
        \begin{itemize}
            \item external labels jumped to
            \item static data
        \end{itemize}
    \end{enumerate}
\end{description}

\section{Linker}
Linker combines several \texttt{.o} files and resolves absolute addresses.
\begin{itemize}
    \item Input: Object code files with information tables (\texttt{foo.o})
    \item Output: Executable code with text and data (\texttt{a.out})
\end{itemize}

\subsection{Three Types of Addresses}
\begin{enumerate}
		\item \texttt{PC}-relative addressing: NEVER relocate
			\begin{itemize}
				\item \texttt{beq}, \texttt{bne}, \texttt{jal}
				\item \texttt{lla}: \texttt{auipc}, \texttt{addi}
			\end{itemize}
		\item Absolute/external function addresses: always relocate
			\begin{itemize}
				\item \texttt{la}: \texttt{auipc}/\texttt{lw} + \texttt{jalr}
			\end{itemize}
		\item Static data references: always relocate
			\begin{itemize}
				\item \texttt{lui}/\texttt{addi}
			\end{itemize}
\end{enumerate}

\section{Loader}
Loader loads executable into memory and runs program.
\begin{itemize}
    \item Input: Executable code with text and data (\texttt{a.out})
    \item Output: Program is run
\end{itemize}

\subsection{Loader Steps}
Loader is the OS.
\begin{enumerate}
    \item Reads executable file’s header to determine size of text and data segments
    \item Creates new address space for program large enough to hold text and data segments, along with a stack segment
    \item Copies instructions and data from executable file into the new address space
    \item Copies arguments passed to the program onto the stack
    \item Initializes machine registers
    \item Jumps to start-up routine that copies program’s arguments from stack to registers and sets the PC
\end{enumerate}
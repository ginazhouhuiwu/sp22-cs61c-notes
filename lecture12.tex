\chapter{RISC-V Datapath}

\section{State Required by RISC-V}
\begin{description}
	\item[Registers:] \texttt{Reg} holds 32 32-bit registers.
	
	\begin{itemize}
	    \item \texttt{rs1} field is always the first register
	    \item \texttt{rs2} field is always the second register
	    \item \texttt{rd} field stores the destination
	    \item \texttt{Reg[0]} is always 0 (writes are ignored)
	\end{itemize}

	\item[Program Counter:] \texttt{PC} holds address of current instruction.
	
	\item[Memory:] \texttt{MEM} holds both instructions and data in a 32-bit byte-addressed memory space
    
    \begin{itemize}
        \item \texttt{IMEM} holds instructions and is \emph{read-only}
        \item \texttt{DMEM} holds data and is accessed by load/store
    \end{itemize}
\end{description}

\section{One-Instruction-Per-Cycle RISC-V Machine}
\begin{align*}
\begin{pmatrix}
\texttt{PC} \\
\texttt{IMEM} \\
\texttt{Reg[]}\\ 
\texttt{DMEM}\\
\end{pmatrix}
\xrightarrow{\text{clock rises}}{}
\left(\texttt{next state}\right)
\end{align*}

On every tick of the clock, the processor executes one instruction.

\begin{enumerate}
    \item Current state outputs drive the inputs to the combinational logic, whose outputs settles at the values of the state before the next clock edge.
    \item At the rising clock edge, all the state elements are updated with the CL outputs and execution moves to next the clock cycle.
    \item Separate instruction/data memory. Memory is asynchronous (unclocked) read but synchronous (clocked) write.
\end{enumerate}

\subsection{Phases of Instruction Execution}
\begin{enumerate}
	\item IF: Instruction fetch (\texttt{DMEM})
	\item ID: Decode/register read (\texttt{IMEM})
	\item EX: Execute (ALU)
	\item MEM: Store memory (\texttt{DMEM})
	\item WB: Register write (\texttt{REG[]})
\end{enumerate}

\section{Instruction Datapaths}
\subsection{\texttt{add} instruction}
We need
\begin{align}
\texttt{Reg}\left[\texttt{rd}\right] &= \texttt{Reg}\left[\texttt{rs1}\right] + \texttt{Reg}\left[\texttt{rs2}\right] \\
\texttt{PC} &= \texttt{PC} + 4
\end{align}

\begin{enumerate}
	\item \texttt{PC} read and incremented.
	\item \texttt{IMEM} dereferenced by \texttt{PC}.
	\item Instruction passed to destination and source registers, and control logic for deciding instruction (signal to \texttt{Reg[]}).
	\item \texttt{rs1}, \texttt{rs2}, \texttt{rd} dereferenced in \texttt{Reg[]}.
	\item ALU performs add logic.
	\item Control logic writes (\texttt{RedWriteEnable}) the result and \texttt{Reg[]} mutates.
\end{enumerate}
The control line that decodes the instructions signals RW-enable in the register and signals the ALU to do the right operation. Most R-format instructions can be implemented this way, depending on what the ALU hears from the control line.

\subsection{More Instructions}
\begin{description}
\item [\texttt{addi}]
From the \texttt{add} datapath, add an \emph{immediate generator} that sign extends the 12-bit immediate. Add a \texttt{mux} at ALU input to select from \texttt{rs2} or \texttt{Imm Gen}.
\begin{itemize}
    \item This modified datapath design works for all I-format instructions.
\end{itemize}

\item[\texttt{lw}]
From the \texttt{addi} datapath, add a path through \texttt{DMEM} before the last step (signaled read-enable by control line) and mux the path by a control line switch.

\item[\texttt{sw}]
Add a path from \texttt{rs2} to \texttt{DMEM}, which is now getting signaled for \emph{write}. Control line signals 0 to register write.
\end{description}

\subsection{Implementing Branches}
B-format is mostly the same as S-format, but we need to add the ability to compare.
\begin{enumerate}
	\item Register comparator mux to control line.
	\item PC mux to ALU.
	\item ALU output mux to PC, signaled by control line.
\end{enumerate}

\subsection{\texttt{jal} instruction}

\chapter{Operating Systems}

What is an operating system?
\begin{enumerate}
	\item Provides \emph{interaction} with the outside world (i.e. devices like mouse, display, network); masks physical resources and limitations
	\item Provides \emph{isolation} between running processes; manages the protection and sharing of resources.
	\begin{enumerate}
	    \item \emph{Context switch}: switches from executing one program to another.
	    \item \emph{Protection}: isolates processes from each other while allowing multiple processes to run on the same processor (time-sharing).
		\item Resources are multiplexed between applications.
	\end{enumerate}
\end{enumerate}

\section{Dual Mode Operation}

Hardware provides at least two modes: kernel (supervisor) mode and user mode. OS mostly runs in user mode, but there are few ways we switch between user and kernel mode.

\subsection{System Calls}
Syscall allows the program to request a service from the OS. Similar to function calls but is executed by the kernel.

\subsection{Interrupts and Exceptions}
\begin{description}
    \item[Interrupts:] asynchronous signals caused by an event \emph{external} to the current running program (e.g. key press); does not need to be handled immediately
    \item[Exceptions:] synchronous errors caused by an event \emph{during} the execution of the current running program (e.g. illegal instruction, divide by zero); must be handled immediately
\end{description}

The \emph{trap handler} services interrupts and exceptions. Prior to execution of the trap handler, all instructions before the faulting instructions must complete and all instructions after must be flushed. 
The trap handler does the following:
\begin{enumerate}
    \item Save the state of current program: save all registers
    \item Determine what caused the interrupt or exception
    \item Selects one of two actions:
    \begin{enumerate}
        \item Continue execution of program (likely interrupts):
        \begin{enumerate}
            \item Restore state of program
            \item Return control to program
        \end{enumerate}
        \item Terminate program (illegal instructions):
        \begin{enumerate}
            \item Terminate program
            \item Schedule new program
        \end{enumerate}
    \end{enumerate}
\end{enumerate}

\section{Kernel}
Kernel is the core of the OS that manages resources including scheduling, memory, and I/O. The \emph{shell} is a program that exposes the OS's services.

\section{Miscellaneous}

\subsection{What happens at boot?}
\begin{enumerate}
	\item The BIOS (Basic Input/Output System) runs power-on-self-test (POST) and executes the bootloader.
	\item The bootloader loads  in part of the OS.
	\item OS initializes services, drivers, etc.
	\item Launch a process that waits for an input in a loop.
\end{enumerate}

\subsection{How do you begin executing a program?}
\begin{enumerate}
    \item The loader loads program into memory
    \item Loader sets \texttt{argc} and \texttt{argv}
    \item OS jumps to \texttt{main} and transfers control to the process
\end{enumerate}
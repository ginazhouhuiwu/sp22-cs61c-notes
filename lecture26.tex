\chapter{Warehouse-Scale Computing}

\section{Warehouse-Scale Computers}
Design goals of a WSC:
\begin{itemize}
    \item Ample parallelism
    \item Scale and its opportunities/problems
    \item Operational costs count
\end{itemize}

\section{Power Usage Effectiveness}
PUE = Total Building Power / IT Equipment Power
\begin{itemize}
    \item 1.0 = perfection
\end{itemize}

\section{Request-Level Parallelism}
RLP represents tasks as a set of largely independent user requests. Computation is easily partitioned across different requests and is load balanced at the DNS level.

\section{Data-Level Parallelism}
DLP distributes data across different nodes, which operate on the data in parallel. DLP on WSC supports parallelism across multiple machines.

\subsection{MapReduce}
Simple data-parallel programming model designed for scalability and fault-tolerance. Users specify the computation in terms of a \emph{map} function and a \emph{reduce} function.

\medskip

Underlying runtime system:
\begin{itemize}
    \item Automatically parallelize the computation across large scale clusters of machines
    \item Handles machine failure
    \item Schedule inter-machine communication to make efficient use of the networks
\end{itemize}

\subsection{Spark}
Apache Spark is a fast and general engine for large-scale data processing.




\chapter{Intro, Number Representation (Jan 18, 2022)}

\section{Great Ideas in Computer Architecture}
\begin{enumerate}
    \item Abstraction: Layers of representation/interpretation
    \item Moore’s Law: Designing through trends
    \item Principle of Locality/Memory Hierarchy
    \item Parallelism \& Amdahl's Law (which limits it)
    \item Dependability via Redundancy
\end{enumerate}

\section{Number Representation}

The atom of computer information is a \emph{bit}. Bits can represent anything.
\begin{itemize}
    \item A \emph{nibble} is 4 bits.
    \item A \emph{byte} is 8 bits.
    \item A \emph{word} is 4 bytes.
    \item A \emph{floating point} is 32 bits (single precision) or 64 bits (double precision).
\end{itemize}

\subsection{Unsigned Representation}
\begin{itemize}
    \item Total range: \([0, 2^{n-1}]\)
\end{itemize}

\subsection{Sign and Magnitude}
MSB is the sign.
\begin{itemize}
	\item Range of negative values: \([-(2^{n-1}-1), -0]\)
	\item Range of positive values: \([+0, 2^{n-1}-1]\)
    \item Total range: \([-(2^{n-1}-1), 2^{n-1}-1]\)
    \item Problem: 2 zeros, difficult to represent
\end{itemize}
	
\subsection{One's Complement}
Flip the bits of the positive number to form the negative counterpart.
\begin{itemize}
	\item Total range: \([-(2^{n-1}-1), 2^{n-1}-1]\)
	\item Problem: impractical
\end{itemize}

\subsection{Two's Complement} 
Shift the number line of One's Complement by 1 to obtain just 1 zero.
\begin{itemize}
	\item Negation: flip all bits and add 1
	\item Range of negative values (shifted right by 1 compared to One's Complement): \([-(2^{n-1}), 0]\)
	\item Range of positive values: \([0, 2^{n-1}-1]\)
	\item Total range: \([-(2^{n-1}), 2^{n-1}-1]\)
\end{itemize}

\begin{description}
	\item[Overflow] Computer is unable to represent the value returned.
	
	If signs of operands are equal and sign of result is different from sign of operands, overflow occurs. 
	\begin{itemize}
	    \item 2 positives: if result < 0 \(\implies\) overflow
	    \item 2 negatives: if result > 0 \(\implies\) overflow
	    \item 1 positive, 1 negative: no overflow
	\end{itemize}
\end{description}

\subsection{Bias Encoding}
Interpret the number as an unsigned integer and shift the number by the bias s.t. the lowest value is at zero.
\begin{itemize}
    \item Two's complement bias: \(N = -(2^{n-1}-1)\)
    \item Total range: \([-(2^{n-1}-1), 2^{n-1}]\)
\end{itemize}


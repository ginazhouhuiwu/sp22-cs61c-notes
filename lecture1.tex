\chapter{Intro, Number Representation (Jan 18, 2022)}
\section{Great Ideas in Computer Architecture}
\begin{enumerate}
    \item Abstraction (Layers of Representation/Interpretation)
    \item Moore’s Law (Designing through trends)
    \item Principle of Locality (Memory Hierarchy)
    \item Parallelism \& Amdahl's law (which limits it)
    \item Dependability via Redundancy
\end{enumerate}

The atom of computer information is a \emph{bit}.
A \emph{byte} is 8 bits.
A \emph{word} is 4 bytes.
A FP number is 64 bits.

\subsection{Binary and decimal}
\begin{description}
	\item[binary] \(\left\{0,1\right\}\)-linear sum of \(\ldots64, 32, 16, 8, 2, 1\).
	\item[decimal] Sum of multiples of powers of 10 (familiar).
\end{description}

Convert \(x\) from decimal to binary, recursively
building from right to left
(least significant, second least significant\ldots most significant):
\begin{enumerate}
	\item If \(x\) is 0, stop.
	\item Parity of \(x\) is next most significant bit.
	\item Floordiv \(x \leftarrow x/2\).
	\item Back to first step.
\end{enumerate}

Hexadecimal groups 4 bits at a time.

\section{Number Representation}

\begin{description}
	\item[sign and magnitude] 1 bit for sign, 7 bits for magnitude
	
	How to add \(a\) and \(b\):
	\begin{enumerate}
		\item if \(a > 0\) and \(b > 0\) add.
		\item if \(a > 0\) and \(b < 0\) subtract.
	\end{enumerate}
	Confusion of \(\pm 0\)??
	
	Logic is too complicated.
	\item[(idea) backwards by forwards?] How to add \(-1\) to a 4-bit number?
	
	\begin{tabular}{rr}
		& 0111 \\
		+ &???? \\ \hline
		&10110
	\end{tabular}
	
	If we use \(1101\), the result is correct. Instead of adding \(-x\) in \(\mathbb{Z}/2n\mathbb{Z}\), simply add the positive \(2n - x\) (there is a simple way to do this conversion, showed later). This is consistent for hardware implementations as all you have to do is
	
	\item[two's complement] Every number with a MSB of 1 is treated as negative.
\end{description}

\subsection{How do you convert a normal decimal integer to two's complement?}
\begin{itemize}
	\item \(\text{two's complement}\rightarrow\text{base 1}\): negate all bits and add 1.
	\item \(\text{base 1}\rightarrow\text{two's complement}\): subtract 1 and negate all bits? (not sure\ldots)
\end{itemize}

\subsection{How do you detect overflow?}
Some hardware overflows are not logical overflows.
\begin{itemize}
	\item Signs of operands are equal, \textsc{and}
	\item Sign of result is different from sign of operands.
\end{itemize}

\subsection{Adding numbers of different bit widths}
How could you add a 4-bit number to an 8-bit number?
\begin{itemize}
	\item For the unsigned case, \emph{zero extension} into upper bits.
	\item If the number is signed, \emph{sign extension}: copy the sign bit (0 or 1) into all of the ``missing'' upper bits.
\end{itemize}

\section{C primer}
C is meant to be somewhat similar to the native machine code that it's compiled on. 

(some advantages and disadvantages omitted)

\subsection{C dialects}
\begin{itemize}
	\item GCC supports this:
\begin{minted}{c}
int main(void) {
	const int SZ = 5;
	int a[SZ];
	//  etc.
}
\end{minted}
	\item But the official way is to do this:
\begin{minted}{c}
#define SZ 5
int main(void) {
	int a[SZ];
	//  etc.
}
\end{minted}
\end{itemize}

\subsection{Preprocessor commands}
\begin{itemize}
	\item Comments replaced with single space.
	\item \texttt{\#include "file.h"} inserts all of {inserts all of the file}.
	\item \texttt{\#include <stdio.h>} \ldots
	\item \texttt{\#define M\_PI (3.14159)} to define a constant.
	\item \texttt{\#if}, \texttt{\#endif}, etc.\ for conditional compilation.
	\item \texttt{-{}-save-temps} with GCC to show the \texttt{*.i} preprocessed file.
\end{itemize}

\subsection{Typed variables}
\begin{itemize}
	\item \texttt{int}, \texttt{long}, \texttt{double}, etc.
	\item \texttt{const} variables never change.
	\item \texttt{enum}:
	
	\begin{minted}{c}
		typedef enum {red, green, blue} Color;
		Color pants = green;
		switch (pants) {
			case red:
				// ...
				break;
			// ...
		}
	\end{minted}
	\item \texttt{int} size:\begin{description}
		\item[Python] 32/\(\infty\)
		\item[Java] 32
		\item[C] 16 or 32 or 64
	\end{description}
	
	Maybe helps processor work more efficiently---could all be 32!
	
	\begin{align*}
		\text{short} \leq \text{int} \leq \text{long} \leq \text{long long}
	\end{align*}
	\item No boolean datatype in C.
		\begin{minted}{c}
			typedef int boolean;
			const boolean false = 0;
			const boolean true = 1;
		\end{minted}
		
		0 and \texttt{NULL} are false; everything else is true.
\end{itemize}

\subsection{Functions}
\begin{itemize}
	\item Like Java.
	\item Declare return and argument types.
	\item \texttt{void} means nothing is returned.
	\item Functions \emph{must} be declared before they are used.
\end{itemize}

\subsection{Undefined variables}
\begin{minted}{c}
	randomly_junk_the_stack();
	// ...
	int x;  // x is UNDEFINED
	printf("x = %d\n", x);  // prints jink
\end{minted}

\subsection{\texttt{struct}s in C}
\begin{itemize}
	\item Groups of variables.
	\item Like Java classes, but no methods.
\end{itemize}
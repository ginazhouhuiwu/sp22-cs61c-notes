\chapter{Virtual Memory II}

\section{Memory Access}
With VM, we need to access memory twice each time when accessing a piece of data.
\begin{enumerate}
    \item Read the page table to perform the address translation
    \item Use the translated address to read the data
\end{enumerate}

\subsection{Translation Lookaside Buffer}
Every instruction and data access requires looking up an address and checking bit flags. To make this happen in less than one cycle, use \emph{translation lookaside buffers}. A TLB is a fully associative cache for the page table \((\text{virtual page number})\to(\text{physical page number})\).

\section{Multi-level Page Tables}
Idea: add more indirection. The first level page table must always be in RAM while the second level page tables can be paged out to disk.

\section{Caches and VM}

Physically indexed, physically tagged cache allows for multiple processes to use the same cache. Virtually indexed, virtually tagged caches does not allow this due to synonyms and homonyms.

\begin{description}
    \item[Synonyms:] different virtual addresses can map to the same physical address
    \item[Homonyms:] same virtual address can map to different physical addresses
\end{description}

\subsection{Virtually Indexed, Physically Tagged Cache}
We want to use the physical address to access the cache to avoid synonym and homonym problem but prevent translating the address from being in the critical path. We will convert the page offset bits into index and byte offset.
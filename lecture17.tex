\chapter{Caches III}

\section{Cache Performance}
\begin{itemize}
    \item Hit rate: number of hits / total number of accesses
    \item Miss rate: 1 - hit rate
    \item Hit time: time to access an item on a cache hit
    \item Miss penalty: time to access block after discovering that it is not in the cache on a cache miss
\end{itemize}

\subsection{Average Memory Access Time (AMAT)}
Minimize: AMAT = Hit Time + Miss Rate * Miss Penalty

\subsection{L2 Cache}
Introduce another cache between memory and the processor cache to reduce L1 miss penalty. L3 may be added as an additional layer of safety net.

\subsection{Performance Optimization}
In general, any variable we can change in caches causes a tradeoff between speed and hit rate, though extreme caches suffer in both metrics. We can optimize AMAT with:
\begin{enumerate}
	\item Reduce hit time
	\begin{itemize}
	    \item Decrease cache size/associativity $\implies$ check fewer spots in our cache (faster hits)
	\end{itemize}
	\item Reduce miss rate
	\begin{itemize}
	    \item Increase cache size
	\end{itemize}
	\item Reduce the miss penalty by using multiple cache levels.
\end{enumerate}

\section{Multi-level Cache Hierarchy}
\subsection{Miss Rate}
\begin{description}
	\item[Local] = \# hits at this level / \# accesses to this level
	\item[global] = \# hits at this level / \emph{total} \# accesses
\end{description}
It is wise to sacrifice L1 miss rate for L1 hit time, as there is more cache (e.g.~L2) to back those up.

Recurrence relation in analyzing multi-level caches:
\begin{description}
    \item[AMAT] = L1 hit time + L1 miss rate * L1 miss penalty
    \item[L1 miss penalty] = L2 hit time + L2 miss rate * L2 miss penalty
    \item[AMAT] = L1 hit time + L1 miss rate * (L2 hit time + L2 miss rate * L2 miss penalty)
\end{description}
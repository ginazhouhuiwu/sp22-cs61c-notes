\chapter{Functional Units, FSMs}

\section{State}
State elements are used to
\begin{itemize}
    \item Store values for some indeterminate amount of time (e.g. registers, memory) 
    \item Control the flow of information between combinational logic blocks
\end{itemize}

\section{Register}
A register is a state element consisting of 32 flip-flips (stores 32 bits).

\subsection{D Flip-Flop}
The output flips and flops between 0 and 1. D is "data", Q is "output".

\subsection{Timing of Flip-Flop}
\begin{description}
    \item[Edge-Triggered D Flip-Flop:] The information storage from D to Q can be \emph{rising edge triggered} or \emph{falling edge triggered}.
    
    \item[CLK:]  Steady square wave that synchronizes everything (i.e. heartbeat of the system).
    
    \item[Clk-to-q Delay:] Time it takes for input to propagate to output after clock trigger $implies$ delay between the triggering edge and output change.
    
    \item[Setup Time:] Time (negative) \underline{before} the triggering edge when input must be stable.
    
    \item[Hold Time:] Time \underline{after} the triggering edge when input must be stable.
\end{description}

\section{Clock Signals}
Clock cycle time must be small enough that inputs to registers don't change within the hold time and large enough to account for clk-to-q times, setup times, and combinational logic delays. When we \emph{overclock} a system, the period shrinks and instability may occur between setup and hold time.

\begin{description}
    \item[Maximum allowable hold time] = clk-to-q delay (starting reg) + shortest CL delay
    
    Intuition: The ending register can hold (remain stable) until the earliest possible time its input might change.
    
    \item[Minimum clock cycle time] = clk-to-q (starting reg) + longest CL delay + setup time (ending reg)
    
    Intuition: Everything between 2 registers must be finished within clock period, thus the bottleneck is the longest path between them.
\end{description}


\section{Combinational Logic}
\begin{itemize}
    \item CL circuits map the present state (PS) input to the next state (NS) output and are the hardware for FSMs.
\end{itemize}

\subsection{Boolean Algebra}
\begin{description}
    \item[DeMorgan's Law:] $\overline{x \cdot y} = \overline{x} + \overline{y}$; \quad $\overline{x + y} = \overline{x} \cdot \overline{y}$
    \begin{itemize}
        \item Intuition: The overline on LHS distributes over all 3 components (inverts x, y, and the logical operator).
    \end{itemize}
    
    \item[Idempotent Law:] $x \cdot x = x$; \quad $x + x = x$
    
    \item[Uniting Theorem:] $xy + x = x$; \quad $(x+y)x = x$
\end{description}

\subsection{Canonical Form}
\begin{enumerate}
    \item Find all rows in truth table where the output is 1.
    \item Write sum-of-products (ORs of ANDs) with output 1.
    \item Simplify using boolean algebra.
\end{enumerate}

\section{Combinational Logic Blocks}

\subsection{Data Multiplexor}
How do we build a 1-bit-wide mux?
\begin{itemize}
    \item $c = \overline{s}a + sb$
\end{itemize}
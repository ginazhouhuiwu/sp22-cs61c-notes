\chapter{C Intro - Pointers, Arrays, Strings}
\section{Pointers in C}
\begin{minted}{c}
int a;
a = -85;
printf("%d", a);
\end{minted}
\ldots yields the following (\emph{Type} and \emph{Name} are just for notation):

\noindent
\begin{tabular}{llrr}
	Type & Name & Addr & Value\\ \hline
	\vdots & \vdots & \vdots & \vdots \\
	\texttt{int} & \texttt{a} & 100 & -85 \\
	\vdots & \vdots & \vdots & \vdots \\
\end{tabular}

A pointer is a way of storing an memory address.

\begin{minted}[tabsize=4]{c}
int *x;     // x is an address of a int
int y = 9;  // y is an int
x = &y;     // x points to address of y
            // "reference" operator
int z = *x; // z gets the value of what x pointed to
            // "dereference" operator
*x = -7;    // assign -7 to x's target.
\end{minted}

\subsection{Generic pointer}
\begin{itemize}
	\item Points to \emph{any object}.
	
	\item \mint{c}{void *vp;}
\end{itemize}

\subsection{Pointer to struct}
\begin{minted}{c}
typedef struct { int x, y; } Point;

// initialize point object
Point pt = { 0, 5 };

// declare pointer
Point pt_ptr = &pt;

// access elements
(*pt_ptr).x = (*pt_ptr).y;

// but alternative syntax is cooler
pp->x = pp->y;
\end{minted}

\subsection{Pointers as function arguments}
\begin{minted}{c}
void f(int x, int *p) {
	// does nothing outside
	x = 5;
	// changes outside!!
	*p = -9;
}

// e.g.
int a = 1, b = -3;
f(a, &b);
// now a = 1 but b = -9.
\end{minted}

\section{Arrays}
\begin{itemize}
	\item To allocate:
		\begin{minted}[autogobble]{c}
		// allocate space
		// unknown content
		int a[5];
		
		// can also initialize at the same time
		int b = { 3, 2, 1 };
		\end{minted}
	\item Array elements are matched sequentially in memory.
	\item C doesn't check array bounds; if you read too far you'll get junk.
	\item Define array size using a constant! (not a literal)
	\begin{itemize}
		\item Bad:
			\begin{minted}[autogobble]{c}
				int i, ar[10];
				for(i = 0; i < 10; i++);
			\end{minted}
		\item Good:
			\begin{minted}[autogobble]{c}
			const int ARRAY_SIZE;
			int i, ar[ARRAY_SIZE];
			for(i = 0; i < ARRAY_SIZE; i++);
			\end{minted}
	\end{itemize}
	\item \texttt{sizeof} works on bounds!
		\begin{minted}[autogobble]{c}
			int array[5];
			// evaluates to 20:
			sizeof(array);
		\end{minted}
		
		but can be awkward on a \texttt{struct} because of padding:
		\begin{minted}[autogobble]{c}
			// should take up 2 + 2 bytes right???
			struct { short a; char c; } s;
			// evaluates to 4!
			sizeof(s);
		\end{minted}
\end{itemize}

\section{Pointer arithmetic}
\subsection{\texttt{char} example}
\begin{minted}{c}
#include <stdio.h>

int main(void) {
	char c[] = { 'a', 'b' };
	
	// c[] already is a pointer! (char[] = char*)
	char *pc = c;
	
	// advance pc to the next location in the array
	pc++;
	
	/*
		(pc - c) is 1.
	*/
	
	int i[] = { 10, 20 };
	int *pi = i;
	pi++;
	/*
		(pi - i) is 4.
	*/
}
\end{minted}
Pointer value is in bytes, but pointer arithmetic is in units of \texttt{sizeof(pointed\_to\_type)}.

\subsection{Array name/pointer duality}
\texttt{char *pstr} and \texttt{char astr[]} are about the same, except \texttt{astr++} is illegal (as though it were constant).
The reason you can't mutate the array reference variable is that it's not a ``real'' variable but just a compiler name.

\subsection{Arrays as arguments}
\emph{An array is always passed as a pointer!} You need to pass in the size of the array, as \texttt{sizeof} will just evaluate to the size of the pointer representation (e.g.~32/64 bits).
\chapter{C Intro - Pointers, Arrays, Strings}

\section{Pointers}
Consider memory to be a single huge array, in which each cell stores some value.

\begin{itemize}
    \item An \emph{address} refers to a particular memory location, represented by signed numbers.
    \item A \emph{pointer} is a variable that contains the address of a variable.
    \item Advantages: Faster \& more efficient to pass a pointer than a block of data; cleaner \& more compact code; low-level access
    \item Disadvantages: Extremely buggy (especially with dynamic memory management); memory leaks
\end{itemize}

\subsection{C Pointer Dangers}
\begin{itemize}
    \item \underline{WARNING}: Declaring a pointer allocates space to hold the pointer, it does NOT allocate the thing being pointed to.
    \item Local variables in C are NOT initialized \(\rightarrow\) may contain garbage
\end{itemize}

\subsection{Pointer Syntax}
\begin{minted}[tabsize=4]{c}
    int *p;         // p is an address of a int
    int x = 3;      // x is an int
    p = &x;         // p points to address of x
    int y = *p;     // y gets the value p points to
    *p = 5;         // update the memory location p points to (x = 5)
\end{minted}
\begin{itemize}
    \item \& = "reference" operator (grab the address)
    \item * = "dereference" operator (follow the pointer)
\end{itemize}

\subsection{Pointers to \texttt{struct}}
\begin{minted}{c}
    typedef struct { int x, y; } Point;
    Point p;        // initialize point object
    Point *paddr;   // declare pointer
    
    /* dot notation */
    int h = p1.x;
    
    /* arrow notation */
    int h = paddr->x;
    int h = (*paddr).x;
\end{minted}

\subsection{Casting}
Casting pointers only changes how they are interpreted.
\begin{minted}{c}
    void foo(void *v) {
        ((Point *) v)->x = 10;
    }
\end{minted}

\begin{itemize}
    \item Treat \texttt{v} as a pointer to a \texttt{Point} struct, sets the value of the \texttt{x} field (pointed to by \texttt{v}) to 10.
    \item If \texttt{v} is a pointer to some other type of data \(\implies\) undefined behavior.
\end{itemize}

\subsection{Pointers to Functions}
\begin{minted}{c}
    int (*fn) (void *, void *) = &foo
\end{minted}

\begin{itemize}
    \item \texttt{fn} initially points to the function \texttt{foo}; it is a function that accepts two void * pointers and returns an int
    \item \texttt{(*fn)(x, y)} calls the function
\end{itemize}

\subsection{Pointer Arithmetic}
Implicitly adds \texttt{1 * sizeof(type)} to the memory address.
\begin{itemize}
    \item \texttt{sizeof()} Operator: Compile time operation that returns number of bytes in object.
\end{itemize}

\subsection{Pointers as function arguments}
\begin{minted}{c}
    void f(int x, int *p) {
    	x = 5;      // does nothing outside
    	*p = -9;    // changes outside!
    }

    int a = 1, b = -3;
    f(a, &b);       // now a = 1 but b = -9.
\end{minted}

\section{Arrays}
\begin{itemize}
    \item An array name is NOT a variable, but a \emph{pointer} to the 0th element.
    \item Maintain a single source of truth: avoid multiple copies of literals by defining array size as a constant.
    \begin{minted}{c}
    // Bad pattern
    int i, ar[10];
    for(i = 0; i < 10; i++){ ... }
    
    // Better pattern
    const int ARRAY_SIZE = 10;
    int i, a[ARRAY_SIZE];
    for(i = 0; i < ARRAY_SIZE; i++){ ... }
    \end{minted}
	\item C doesn't check array bounds: if you read too far you will get garbage.
\end{itemize}

\subsection{Arrays as arguments}
An array is always passed as a \emph{pointer} You need to pass in the \emph{size} of the array, as \texttt{sizeof} will just evaluate to the size of the pointer representation (e.g.~32/64 bits).

\subsection{Array Name/Pointer Duality}
\texttt{char *pstr} and \texttt{char astr[]} are about nearly the same, except \texttt{astr++} is illegal (as though it were constant). We cannot mutate the array reference variable is that it is not a variable but just a compiler name.
\begin{itemize}
    \item \texttt{ar[32]}: array variable with 32 elements, works like a pointer
    \item \texttt{ar[0]}: same as \texttt{*ar}
    \item \texttt{ar[i]}: same as \texttt{*(ar+i)}
\end{itemize} 

\subsection{Arrays, Structures, and Pointers}
\begin{minted}{c}
    typedef struct bar {
     char *a;       // A pointer to a character 
     char b[18];    // A statically sized array of characters
    } Bar;
    ...
    Bar *b = (Bar*) malloc(sizeof(struct bar));
    b->a = malloc(sizeof(char) * 24);
\end{minted}

\begin{itemize}
    \item Requires 24 bytes on a 32b arch: 4 bytes (pointer a) + 18 bytes (18-char array b) + 2 bytes (padding)
\end{itemize}

Examples:
\begin{itemize}
    \item \texttt{b->b[5] = 'd'}
    \begin{itemize}
        \item \texttt{b} field starts on the 4th byte
        \item Location written to the 10th byte pointed to by \texttt{b}: \\
        \texttt{*((char *) b + 4 + 5) = 'd'}
    \end{itemize}
    
    \item \texttt{b->a[5] = 'c'} 
    \begin{itemize}
        \item Location written to the first word pointed to by \texttt{b} treated as a pointer and advanced by 5: \\ \texttt{*(*((char **) b) + 5) = 'c'}
    \end{itemize}
    
    \item \texttt{b->a = b->b} 
    \begin{itemize}
        \item Location written to the first word pointed to by \texttt{b}: \\
        \texttt{*((char **)b) = ((char *) b) + 4}
    \end{itemize}
\end{itemize}

\section{C Strings}
A string in C is simply an array of characters, where the last character is followed by a \emph{null terminator} (0 byte).

\subsection{Arguments to \texttt{main}}
\begin{minted}{c}
int main(int argc, char * argv[]);
\end{minted}
\begin{itemize}
	\item \texttt{argc} is the \emph{number} of commands
	\item \texttt{argv} points to an array of strings.
		\begin{itemize}
			\item pointer to an array of pointers!
			\item \texttt{argv[0]} is the name of the executable.
		\end{itemize}
\end{itemize}
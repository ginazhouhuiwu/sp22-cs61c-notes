
\chapter{Floating Point}

\section{IEEE-754 Floating Point Standard}
So far we can represent \(2^N\) integers, but what about very small and very large numbers? The goal of establishing a floating point standard is to keep \emph{high precision} and keep a decimal encoding that is compatible with two's complement.

\subsection{Floating Point Components (32 Bits)}
\begin{itemize}
    \item Sign (1 bit): 0 means positive, 1 means negative
    \item Exponent (8 bits): Biased notation $\implies$ smallest number is written as all zeros
    \begin{itemize}
        \item Bias = $-(2^{N-1}-1)$
    \end{itemize}
    \item Significand/Mantissa (23 bits): In normalized form, the implicit 1 on the left of the binary point is not stored
\end{itemize}

\subsection{Floating Point Conversion}
\begin{description}
    \item[Normalized Floats:] \(\text{Value} = (-1)^{\textit{Sign}} * 2^{\textit{Exponent + Bias}} * 1.\textit{Significand}_2\)
    \item[Denormalized Floats:] \(\text{Value} = (-1)^{\textit{Sign}} * 2^{\textit{Exponent + Bias + 1}} * 0.\textit{Significand}_2\)
\end{description}

\subsection{Range of Floating Point Values}
\begin{tabular}{ |c|c|c| } 
 \hline
 \textbf{Exponent} & \textbf{Significand} & \textbf{Meaning} \\ 
 \hline
 0 & 0 & \text{Zero} \\
 0 & \text{Nonzero} & \text{Denorm} \\
 1-254 & \text{Anything} & \text{Normal} \\
 255 & 0 & \text{Infinity} \\
 255 & \text{Nonzero} & \text{NaN} \\
 \hline
\end{tabular}

\begin{itemize}
    \item Positive range: \([2^{-126}, (2 - 2^{-23}) \times 2^{127}\)
    \item Negative range: \([-(2 - 2^{-23}) \times 2^{127}, -2^{-126}]\)
    \item Overflow: Magnitude of value is too \emph{large} to represent.
    \item Underflow: Magnitude of value is too \emph{small} to represent.
\end{itemize}

\begin{description}
    \item[Smallest positive value (denorm):] exponent = 0; significand = all 1s
    
    \((-1)^0 \times 0.0000....0001 \times 2^{0 - 127 + 1} = 2^{-23} \times 2^{-126}\)
    
    \item[Smallest positive norm value:] exponent = 1; significand = last bit 1
    
    \((-1)^0 \times 1.0000....0000 \times 2^{1-127} = 2^{-126}\) 
    
    \item[Largest positive value (normal):] exponent = 254 (all 1s); significand = all 1s
    
    \((-1)^0 \times 1.1111...1111 \times 2^{254 - 127} = (1 + (2^{-1} + 2^{-2} + ... + 2^{-23}) \times 2^{127}) = (1 + (1 - 2^{-23}) \times 2^{127})\)
    
    \item[Largest positive denorm value:] exp = 0; significand = all 1s
    
    \((-1)^0 \times 0.1111...1111 \times 2^{0 - 127 + 1} = (1 - 2^{-23}) \times 2^{-126}\)
\end{description}

\section{Floating Point Step Size}
Let $x$ be the biased exponent and $y$ be the significand. 
\begin{itemize}
    \item Current number = \((1+y) \times 2^{x - 127}\)
    \item Next number = \((1+y + 2^{-23}) \times 2^{x - 127}\)
    \item Step size = next\_num - curr\_num = \(2^{x-150}\)
\end{itemize}

Notice there are all \(2^{23} ~ 8 \text{ million}\) possible bit patterns with \emph{equal} spacing between neighboring powers of 2. The step increases by a factor of 2 for every time the exponent increases by 1.

\begin{description}
    \item[Denorm Step Size:] Equivalent for all denorm values since they all have the same exponent.
    \begin{itemize}
        \item Current number = \(y \times 2^{-126}\)
        \item Next number = \((y + 2^{-23}) \times 2^{-126}\)
        \item Step size = \(2^{-149}\)
    \end{itemize}
\end{description}

\section{Floating Point Caveats}
\begin{description}
    \item[Non-associative:] FP \emph{approximates} the real result.
    \begin{itemize}
        \item e.g. \(x = -1.5 \times 10^{38}\), \(y = 1.5 \times 10^{38}\), \(z = 1.0\)
        \item \(x + (y + z) = -1.5 \times 10^{38} + (1.5 \times 10^{38} + 1.0) = -1.5 \times 10^{38} + 1.5 \times 10^{38} = 0.0\)
        \item \((x + y) + z = (-1.5 \times 10^{38} + 1.5 \times 10^{38}) + 1.0 = 0.0 + 1.0 = 1.0\)
    \end{itemize}
    
    \item[Precision vs. Accuracy:] \emph{Precision} is the count of bits used to represent a value; \emph{accuracy} is the different between the actual value and its computer representation.
    
    \item[Rounding:] May occur during calculation and conversion/casting.
\end{description}
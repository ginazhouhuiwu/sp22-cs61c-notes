\chapter{Assembly Intro - RISC-V}

\section{Assembly Language}

The basic job of a CPU (Central Processing Unit) is to execute \emph{instructions} sequentially. The set of instructions that a particular CPU can execute is called its Instruction Set Architecture (ISA).

\subsection{RISC-V (Reduced Instruction Set Computer)}
A single instruction can only perform one operation $\implies$ keeps the instruction set small and simple for faster hardware.

\section{Registers}
Assembly languages use \emph{registers} rather than \emph{variables}. Registers are small storage units inside the CPU. Operations can only be performed on these and they are extremely fast because they are built directly in hardware.
\begin{itemize}
    \item RISC-V has 32 registers (32-bit variant) labeled \texttt{x0}--\texttt{x31}.
    \item Each register stores 4 bytes (128 bytes total).
    \item Groups of 32 bits is a \emph{word}.
    \item Unlike variables, registers do not have type.
\end{itemize}



\section{RISC-V Instructions}
Syntax is rigid: Operator $\rightarrow$ destination register $\rightarrow$ 2 source registers.

\subsection{Immediates}
\begin{enumerate}
	\item Immediates are numerical constants.
	\begin{minted}[autogobble]{asm}
	    addi x3, x4, -10  # f = g - 10
	\end{minted}
	\item Register zero \texttt{x0} is hard-wired to value 0.
	\begin{minted}[autogobble]{asm}
	    add x3, x4, x0  # f = g
	\end{minted}
\end{enumerate}

\subsection{Data Transfer}
\begin{itemize}
    \item We \emph{load from} memory to register and \emph{store to} memory from register (left to right).
    \item Memory addresses are in \emph{bytes} $\implies$ word addresses are 4 bytes apart.
    \item Little-endian convention: Word address is the same as address of rightmost (least-significant) byte.
\end{itemize}

\begin{description}
    \item[Load From:] $\texttt{lw dest\_reg offset(base\_reg)}$
    \begin{itemize}
        \item Data flow is right to left.
    \end{itemize}
    \begin{minted}[autogobble]{c}
		int A[100];
		g = h + A[3];
	\end{minted}
		$\qquad \downarrow$
	\begin{minted}[autogobble]{asm}
		lw x10, 12(x13)  # reg x10 gets A[3] (A + 4 words)
		add x11, x12, 10 # g = h + A[3]
	\end{minted}
	
	\item[Store To:] $\texttt{sw dest\_reg offset(base\_reg)}$
	\begin{itemize}
        \item Data flow is left to right.
    \end{itemize}
    \begin{minted}[autogobble]{C}
		int A[100];
		A[10] = h + A[3];
	\end{minted}
		$\qquad \downarrow$
	\begin{minted}[autogobble]{asm}
		lw x10, 12(x13)    # temp reg x10 gets A[3]
		add x10, x12, x10  # temp reg x10 gets h + A[3]
		sw x10, 40(x13)    # A[10] = h + A[3]
	\end{minted}
	
	\item[Loading and Storing Bytes:] $\texttt{lb}$ and $\texttt{sb}$ transfer data at the byte level.
	\begin{itemize}
	    \item When a byte is loaded from memory, it is placed into the lowest byte of the destination register and sign extended.
	    \item When a bite is stored, only the lower 8 bits of the register is copied to memory and there is no sign extension.
	\end{itemize}
\end{description}

\subsection{Branch Instructions (Decision Making)}
We use \emph{labels} to give control flow instructions places to go.
\begin{enumerate}
    \item Condition branch: Only branch if some condition is met (e.g. $\texttt{beq}$, $\texttt{bne}$, $\texttt{blt}$, $\texttt{bge}$).
    \item Unconditional branch: Always branch (e.g. $\texttt{j label}$)
\end{enumerate}

\subsection{Loop example}
Assume \texttt{x8} holds the address of the array.
\begin{multicols}{2}
	\begin{minted}{c}
    int A[20];
    int sum = 0;
    for (int i = 0; i < 20; i++)
    	sum += A[i]
	\end{minted}
	
	\columnbreak
	
	\begin{minted}{asm}
	addi x9, x8, 0    # x9 = &A[0]
	addi x10, x0, 0   # sum = 0
	addi x11, x0, 0   # i = 0
	addi x13, x0, 20  # x13 = 20
Loop:
    bge x11, x13, Done
	lw x12, 0(x9)     # x12 = A[i]
	add x10, x10, x12 # sum += A[i]
	addi x9, x9, 4    # x9 = &A[i++]
	addi x11, x11, 1  # i++
	j Loop
Done:
	\end{minted}
\end{multicols}

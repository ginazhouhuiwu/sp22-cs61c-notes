\chapter{C Memory Management (Jan 27, 2022)}

\section{Endianness}
The endianness of a system decides how multibyte blocks are stored in memory.
\begin{minted}{c}
    union confuzzle { int a; char b[4]; };
    union confuzzle foo;
    foo.a = 0x12345678;
\end{minted}
In a 32b architecture, what would \texttt{foo.b[0]} be?
\begin{itemize}
    \item Big-endian: \emph{Most} significant byte of a word at the smallest memory address (first character): \texttt{0x12}.
     
    \item Little-endian: \emph{Least} significant byte of a word at the smallest memory address: \texttt{0x78}.
\end{itemize}

\section{Memory Allocation}
\begin{tabular}{ll}
	Address & Contents\\\hline
	FFFF\,FFFFF & stack (grows down)\\
	\(\vdots\) & \(\vdots\) \\
	? & heap (grows up) \\
	? & static data (mutable) \\
	0000\,0000 & code (immutable)
\end{tabular}
\begin{description}
	\item[code] does not change.
	\item[static data] loaded when program starts, \emph{can} be modified.
	\item[stack] local args, allocated when function is called, grows downward (from higher to lower addresses).
		\begin{itemize}
			\item \emph{stack pointer} is a register that moves down with deepening calls to track where the last frame is
		\end{itemize}
	\item[heap] space for dynamic data, allocated and freed as needed.
\end{description}

\section{Heap}
\begin{description}
    \item[\texttt{void *malloc(size\_t n)}] Allocate a block of \emph{uninitialized} memory.
    \begin{itemize}
        \item Note: Subsequent calls probably will not yield adjacent blocks.
    \end{itemize}
    
    \item[\texttt{void free(void *p)}] \texttt{p} is a pointer containing an address originally returned by \texttt{malloc()}.
    
    \item[\texttt{void *calloc(size\_t nmem, size\_t size)}] Allocates \texttt{(nmem * size)} bytes and initialize all memory in it to 0.
    
    \item[\texttt{void *realloc(void *p, size\_t size)}] Resize a previously allocated block at \texttt{p} to a new size.
    \begin{itemize}
        \item Will NOT update other pointers pointing to the same block of memory.
    \end{itemize}
\end{description}

\subsection{Strings}
We can copy a string by finding the length, allocating a new array, and calling strcpy.
\begin{minted}{c}
    /* add 1 for null terminator 
    since strlen does not include it */
    a = malloc(sizeof(char) * (strlen(b) + 1) ); 
    strcpy(a, b) or strncpy(a, b, strlen(b) + 1); 
\end{minted}


\section{Stack}
Contents of a stack frame:
\begin{itemize}
			\item function arguments
			\item local variables
			\item who called me? (for a return)
		\end{itemize}
		
\section{Alignment}
Default alignment rules centered around a 32b architecture:
\begin{itemize}
    \item char: 1 byte, unaligned
    \item short: 2 bytes, 1/2 world aligned (0, 2, 4, ...)
    \item int, pointers: 4 bytes, word aligned
\end{itemize}

\section{Bit Operations}
\begin{itemize}
    \item \(\mid\) : turns bits ON
    \item \& : turns bits OFF
    \item \^ \space \space: FLIPS bits
    \item a << b : shift value of a to the left by b bits \(\implies\) multiplying by \(2^b\)
    \item a >> b : shift value of a to the right by b bits \(\implies\) NOT quite the same as dividing by \(2^b\) due to rounding
    \begin{itemize}
        \item If a is signed, sign extend (copy MSB)
        \item If a is unsigned, zero extend
    \end{itemize}
\end{itemize}

\subsection{Basic Operations}
\begin{minted}{c}
    /* Returns the Nth bit of X. */
    unsigned get_bit(unsigned x, unsigned n) {
        return (x >> n) & 1;
    }
    
    /* Clears the Nth bit of the value of x to 0. */
    void clear_bit(unsigned *x, unsigned n) {
        *x &= ~(1 << n);
    }
    
    /* Sets the Nth bit of the value of x to 1. */
    void set_bit(unsigned *x, unsigned n) {
        *x |= (v << n);
    }
    
    /* Flips the Nth bit in X. */
    void flip_bit(unsigned *x, unsigned n) {
        *x ^= (1 << n);
    }
    
    /* Changes the Nth bit of the value of x to v. */
    void change_bit(unsigned *x, unsigned n, unsigned v) {
        *x = (*x & ~(1 << n)) | (v << n);
    }
\end{minted}
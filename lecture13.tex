\chapter{RISC-V Single-Cycle Control}

\section{Controller}
Control is the hardware that tells the datapath what to do.

\subsection{Controller Relization Options}
\begin{description}
    \item[Read-Only Memory (ROM):] Manually programmed to implement the truth table in the control block; cannot be electronically modified.
    \item[Combinatorial Logic:] Uses logic synthesis to convert truth tables to networks of gates.
\end{description}

\section{Instruction Timing}
Approximate worst case instruction timing:
\begin{enumerate}
	\item PC updates
	\item Instruction fetch
	\item Instruction decode
	\item Execute ALU
	\item Read memory
\end{enumerate}

\section{Performance}
\subsection{Performance Measures}
Our RISC-V executes instructions at 1.25 GHz (1 instruction every 800 ps); how can we improve its performance? Less time for each instruction? More instructions per unit time?

\medskip
We take the following into account:
\begin{itemize}
    \item \emph{Latency:} Instruction execution time; fundamental and cannot be changed
    \item \emph{Throughput:} Total number of instructions executed per unit time; can be improved by performing tasks in parallel
    \item \emph{Energy Efficiency:} Energy per instruction
\end{itemize}


\subsection{Iron Law}
\begin{align}
	\frac{\text{time}}{\text{program}} &= 
	\left(\frac{\text{instructions}}{\text{program}}\right)
	\left(\frac{\text{cycles}}{\text{instruction}}\right)
	\left(\frac{\text{time}}{\text{cycle}}\right)
\end{align}

\begin{itemize}
    \item Instructions per program depends on task, algorithm, programming language, compiler, ISA.
    \item Average clock cycles per instruction is 1 on our single-cycle datapath, but can vary depending on ISA, processor implementation (microarchitecture).
    \item Time per cycle (clock frequency) is determined by architecture (e.g.~length of critical path), technology (transistor size), supply voltage.
\end{itemize}
 
\subsection{Energy per Task}
\begin{align}
\frac{\text{energy}}{\text{program}}
&= \left(\frac{\text{instructions}}{\text{program}}\right)
\left(\frac{\text{energy}}{\text{isntructions}}\right)\\
& \propto \frac{\text{instructions}}{\text{program}}
\left(CV_{dd}^2\right)
\end{align}

Capacitance depends on the physical complexity of the processor. We want to reduce capacitance and supply voltage to reduce energy per task.

\subsection{Energy Iron Law}
\begin{align}
    \text{performance} = \text{power} \cdot \text{energy efficiency}
\end{align}

Energy efficiency is the key metric in all computing devices/power-constrained systems.

\subsection{End of scaling}
At low voltage, the \emph{leakage power} where transistor switches don't fully turn off is icnreased.